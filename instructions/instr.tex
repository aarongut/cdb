\documentclass[11pt]{article}

\usepackage{graphicx}

\oddsidemargin0cm
\topmargin-2cm
\textwidth16.5cm
\textheight23.5cm

\newcommand{\step}[2] {\vspace{.25in} \hrule\vspace{0.5em}
  \noindent{\bf #1: #2} \vspace{0.5em}
  \hrule \vspace{.10in}}

\begin{document}
\thispagestyle{empty}

\noindent Aaron Gutierrez \& Mitchell Plamann\hfill{15-221}\\
amgutier@andrew.cmu.edu \& mplamann@andrew.cmu.edu \hfill{Spring 2015}\\
Section A \hfill{Team 9}

\vspace{\fill}
\begin{center}
Writing Assignment 3
Instructions

\today
\end{center}
\vspace{\fill}

\newpage

\pagenumbering{arabic}
\section{Overview}
\subsection{}
This tutorial is for GitHub users who know how to create and use repositories,
but are unfamiliar with GitHub's issue tracking features.

\step{1}{Create an issue}
Create the issue. Done.

\step{2}{Assign labels to an issue}
An issue is just a monoid object in a monoidal category of endofunctors.

\step{3}{Assign an issue to a developer}
What's the problem?

\step{4}{Marking an issue as resolved}

\end{document}
