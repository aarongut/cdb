\documentclass[11pt]{article}

\usepackage{graphicx}

\oddsidemargin0cm
\topmargin-2cm
\textwidth16.5cm
\textheight23.5cm

\newcommand{\step}[2] {\vspace{.25in} \hrule\vspace{0.5em}
  \noindent{\bf #1: #2} \vspace{0.5em}
  \hrule \vspace{.10in}}

\begin{document}
\thispagestyle{empty}

\noindent Aaron Gutierrez \& Mitchell Plamann\hfill{15-221}\\
amgutier@andrew.cmu.edu \& mplamann@andrew.cmu.edu \hfill{Spring 2015}\\
Section A \hfill{Team 9}

\vspace{\fill}
\begin{center}
Writing Assignment 3
Instructions

\today
\end{center}
\vspace{\fill}

\newpage

\pagenumbering{arabic}
\begin{center}
{\Huge Creating and Closing GitHub Issues}
\end{center}
\section{Overview}
GitHub provides a great tool for multi-user collaboration on software projects.
On top of the version control features, GitHub offers sophisticated issue
tracking integrated right in with the repository.  Tracking issues makes it easy
to keep track of bugs, new feature ideas, and feedback from testers, all of
which help keep your project organized.  In the next five steps, you will learn
how to create a new issues, assign an issue to a contributor, and mark an issue
as resolved.

This tutorial is for GitHub users who know how to create and use repositories,
but are unfamiliar with GitHub's issue tracking features. You need to have an
existing GitHub repository, a web browser, and a git client on your computer.
Once you're done, you'll be able add, manage, and close issues for every project
you have on GitHub.

\section{Steps}

\step{1}{Create an issue}
Create the issue. Done.

\step{2}{Assign labels to an issue}
An issue is just a monoid object in a monoidal category of endofunctors.

\step{3}{Assign an issue to a developer}
What's the problem?

\step{4}{Marking an issue as resolved}

\end{document}
