\documentclass[11pt]{article}

\usepackage{setspace}
\usepackage[margin=1in]{geometry}

\usepackage{tabularx}

\usepackage{tocloft}
\renewcommand{\cftsecleader}{\cftdotfill{\cftdotsep}}

\usepackage{txfonts}

%% END PREAMBLE
\begin{document}

%\doublespacing
\setlength{\parindent}{2em}

\begin{titlepage}
\clearpage
\thispagestyle{empty}

Letter of transmittal goes HEER

\end{titlepage}

\begin{titlepage}
\clearpage
\thispagestyle{empty}

This is a title page

Included is an abstract
\end{titlepage}

\pagenumbering{roman}
\tableofcontents
\newpage

\pagenumbering{arabic}

\section{Introduction}
I'd like to introduce some things.

\section{Literature Review}
\par
Here we will discuss projects similar to ours, as well as technology
we plan to use for our project.
\begin{itemize}
\item Building an In-Browser JavaScript VM and Debugger Using Generators\\
  http://amasad.me/2014/01/06/building-an-in-browser-javascript-vm-and-debugger-using-generators/
  \par
  In this blog post, Amjad Masad describes how he implemented
  debug.js, a JavaScript debugger running inside the web
  browser. Since we wish to implement a C0 debugger running inside the
  web browser, Masad's notes seem to be relevant.  Specifically, this
  post discusses the architecture of debug.js, as well as various
  challenges Masad faced in developing it.  Debug.js was designed in
  two separate parts: a virtual machine and a debugger. The virtual
  machine handled the task of evaluating the JavaScript program being
  debugged, adding support for stopping, starting, and analyzing the
  program. The debugger was the visual interface to the virtual
  machine, allowing users to control the virtual machine and see its
  output.
  \par
  Masad also discusses challenges he overcame while writing
  debug.js. These included being able to step line-by-line through a
  program, keeping track of a call stack, handling errors and
  exceptions, implementing native APIs, and dealing with events. While
  many of the details will be different when working with C0, we must
  still consider all of these challenges in developing our project.

\item The Architecture of Open Source Applications (Volume 2): Processing.js\\
  http://www.aosabook.org/en/pjs.html
  \par
  In Chapter 17 of Mike Kamermans' book The Architecture of Open
  Source Applications, he discusses the design of
  Processing.js. Processing is a Java-based programming language
  designed to help teach computer programming in a visual
  context. Processing.js is a project designed to run Processing
  programs in the web browser using only JavaScript.  This was done by
  writing a own Java-to-JavaScript compiler, and running the resulting
  code attached to a HTML canvas.  Along the way, the developers ran
  into several different challenges, mostly due to differences between
  the Java and JavaScript languages.  The largest difference between
  the languages was that JavaScript programs do not get their own
  thread; the browser freezes if a JavaScript program tries to run for
  too long.  We must consider this issue among others for our project.

\item Node.js Documentation\\
  http://nodejs.org/documentation/
  \par
  This is the documentation for the node.js platform.  We plan to use
  node.js to write the server-side code for our project.  We believe
  that node is a good fit for our project since we are writing
  JavaScript for the client side of our code, so this will let us work
  in the same language on the server and client side.  Also, we can
  make use of the existing cc0 compiler to translate C0 source code to
  the bytecode our virtual machine will run. This is the same compiler
  used in 15-122, and integrating it with our server will make it
  feasible to run actual C0 source code.
\end{itemize}
\section{Plan}
\par
Our goal is to build a web application that can debug C0 code.
The user will type in or upload C0 source files.
Once this is done, these files will be transferred to our server,
where the cc0 compiler will be used to
generate bytecode corresponding to the user's source code.
This bytecode will be sent back to the user's web browser,
where we will be running a C0 virtual machine.
The user will be able to control this virtual machine as it executes their code. 
This will give the user the ability to run their code line-by-line,
to set breakpoints, view stack traces, and see the values of variables.
By providing access to all this information,
we hope to make it easier for users to write and debug C0 programs.
\par
For version control, we will use a git repository hosted on GitHub. 
We will use a Gantt chart, shown later in this proposal, to stay on schedule.


\section{Benefits}
\par
This project will benefit students in 15-122 Principals of Imperative
Computation at Carnegie Mellon University by helping them create correct
programs. The C0 Debugger will enable students to understand how their programs
execute and find where problems originate more easily than with existing tools.
In addition to debugging, students will have better knowledge for how the
underlying computation model works when evaluating their code.
\par
The C0 Debugger will also enable students to test simple programs with little
setup, using only a web browser. They will no longer have to set up and become
familiar with a Unix environment before they can program, making C0 accessible
to more people.

\section{Approach}
\par
The approach section contains our methodology, how we plan to implement the project,
and our project schedule, the timeline we plan to adhere to.
The methodology outlines the specific tools we will use to complete the project in a
timely manner whereas the schedule outlines the deadlines by which we hope to have
certain tasks completed.

\subsection{Methodology}
The C0 Debugger is designed for the CMU teaching language, C0.
It will be hosted on <BLANK> with the website itself designed in CSS and HTML,
using Node.js to run most of the core functionality.
We will first deploy a blank template website after which half of the team wil work on
parsing C0 bytecode and the other half will work on creating a meaningful user experience.
Once both teams have made reasonable progress, they will combine the two units to
complete the basic outline of the project.

\subsection{Project Schedule}
The project will be separated into five main phases: Basic Website Design,
Backend implementation, Frontend Implementation, User Testing, and Revisions.
The first phase should take <POSSIBLY CHANGE THIS> less than a week with the
next two phases occurring simultaneously and composing the rest of the month's work.
User implementation and revisions will then hopefully take up the remainder of the
alloted time, with extra time padded in case implementation or revisions are more
extensive than we have predicted.

\section{Evaluation Criteria}
\par
The goal of our website, as mentioned earlier in the proposal,
is to provide a tool for 15-122 students to easily step through their C0 code
as a means of debugging and to gain a deeper level of understanding for the steps
their code is actually taking.
\par
In order to evaluate our final project, we would test the product on various groups of students.
Both those who have completed 15-122 in the past and those currently enrolled.
Unfortunately, due to the time constraints of the project, these students will no
longer actively code in C0 by the time they see our product, but their interactions
with it will still have been recent enough for them to provide meaningful feedback.
With their feedback, we will determine how well our product succeeds at its aforementioned
objectives and plan a series of modifications based on the comments we receive. We will make
sure that the stepping tool and GUI are fully functional before the group testing phase so
that uninformative bugs do not catch the attention of our test subjects, and they instead
provide us with information to improve the user experience as a whole.
\par
Our main goal is to provide these students with a useful debugging tool,
so their feedback is invaluable in slowly modifying our project to better suit their needs.

\section{Qualifications of Team Members}
\par
We are a team of sophomore CS majors who have varied experience in the field.
\par
Suhaas Reddy has had two years of programming experience.
He has also served as a course assistant for the School of Computer Science
for three semesters which gives him an understanding of what computer science students may
need from a debugging tool. This spring Suhaas competed in his first Hackathon where he and a
group of three other students worked to create a webapp which eliminated unwanted Craigslist
postings from view using machine learning, and sorted the rest based on specific attributes.
He is well-versed in Python, C0, and C.
\par
Shyam Raghavan has had seven years of programming experience.
He has served as a teaching assistant for the School of Computer Science for two semesters,
specifically for 15-122, which makes him especially prepared to create a teaching tool for C0,
the main language used in the course.
In the past, Shyam has interned at Thumbtack, a west coast company which specializes in
enabling consumers to hire experience professionals from a variety of fields.
Shyam has experience with C, JavaScript, and C0.
\par
Aaron Gutierrez has had <blank> years of programming experience.
He has also served as a teaching assistant for the School of Computer Science for two semesters,
specifically for 15-122 just like Shyam. Aaron is very well-versed in JavaScript, C, and C0.
\par
Mitchell Plamann has had nine years of programming experience.
He has interned at Rockwell Automation, doing firmware developement for embedded systems.
Mitchell has coded extensively in C, Python, and Haskell. % You know I had to say it, right?

\section{Sources Cited}
TBD

\end{document}
